% A Theory section should extend, not repeat, the background to the
% article already dealt with in the Introduction and lay the
% foundation for further work.

\section{Background}
\label{sec: backg}

\subsection{Contact Theory}
\par Before Allport hypothesized the ``Intergroup Contact Hypothesis" in 1954, it was believed that inter-group contact would inevitably lead to conflict, result from nineteenth century Social Darwinism, where most groups felt superior to others, naturally leading to hostility. However, no studies were conducted at that time and therefore no empirical evidence was found to support the claim. \cite{pettigrew_1998} 
\par Allport, in \textit{The Nature of Prejudice (1954)}, after extensive study, would lately adapt these four conditions into four ``positive factors" that need to be present to reduce prejudice:

\begin{enumerate}
\item \textbf{Equal status between groups.} This equal status that Allport states refers \textbf{within} the situation, not \textbf{coming into}. Some writers defend that should be of equal status prior to entering the situation, but research has shown that equal status \textit{within} is enough and even more important than outside status.
\item \textbf{Common goals.} To reduce prejudice, active inter-group contact must share a common goal. By having the same objectives, the team constituents work effectively, harder and unitedly, as the different groups rely on each other, to accomplish it.
\item \textbf{Inter-group cooperation.} To work in unity towards the common goal, logically, there must not be group competition. Cooperation should be independently emphasized to each subject, as the feeling of competition undermines the effort made by the rest of the team.
\item \textbf{Support of authorities, law or custom.} With a climate of support surrounding the contact's environment, inter-group contact is more readily accepted and has more positive results. Field research in the military, business and religious institutions has emphasized the importance in support by the authority, as it is that authority that establishes the norms of acceptance.
\end{enumerate}

% #############################################################################
\subsection{Augmented Reality}
\par A variation of Virtual Reality, but different enough to warrant a distinction, the term ``Augmented Reality" has been around since the early 1990s. While the former focuses on creating a separate environment from the user point-of-view, isolating him from his ``real" environment, either through visual output, sounds, smells, or even tact, the last tries to enhance the current environment where the user is present, usually capturing it with a camera and overlaying it with information (e.g. text, images) but nevertheless still allowing the user to access the original information captured by his bio-sensors, i.e., eyes, ears, hence ``augmenting" his senses. \cite{sutherland_1968}
\par Azuma describes Augmented Reality as systems that: \cite{azuma_1997}
\begin{itemize}
    \item Combine real and virtual;
    \item Are interactive in real time;
    \item Are registered in three dimensions;
\end{itemize}

\par Augmented Reality has since gained popularity in the game community with the release of PokémonGO. In their work Das et al. hypothesize that Augmented Reality could have a social impact that is not normally associated but is inherent to the genre. Das et al. state that "in comparison to traditional video games, ARGs may be inherently more social.  Players are required to interact with the surroundings and often encounter friends and fellow players. For example, Pokémon GO and other ARGs have many features that promote social interaction between players. (...) Players on the same team are encouraged to work together to strengthen their Pokémon and gain or maintain control of PokéGyms. The team feature further increases the social component of Pokémon  GO  because  players  must  work  with  teammates  in  order  to  advance. " \cite{das_zhu_mclaughlin_bilgrami_milanaik_2017} This is an important foundation because it comes close to what we want to do with LINA, not exactly with the gym or Pokémon mechanic, but more on the part of using Augmented Reality to socialize and advance on the game. 


%####################################################################
\subsection{User-Centered Design}
\par User-Centered Design, is a broad term to describe design processes in which the users affect the development decisions. There are several ways the users can be involved: from requirements gathering and usability testing, to being made partners to designers through the design process.\cite{Abras04user-centereddesign} The term was coined by Donald Norman in his research laboratory in the University of California San Diego, but the concept was formed in his book \textit{The Psychology Of Everyday Things} (1988), where he states four rules for a design to be user-driven:
\begin{itemize}
    \item Make it easy to determine what actions are possible at any moment.
    \item Make things visible, including the conceptual model of the system, alternative actions, and the results of actions. 
    \item Make it easy to evaluate the current state of the system.
    \item Follow natural mappings between intentions and the required actions; between actions and the resulting effect; and between the information that is visible and the interpretation of the system state.
\end{itemize}
\par But for Norman just saying a design should be intuitive is not enough, so he states how important it is to consider some additional design principles to facilitate the designer and the user:
\begin{enumerate}
    \item Simplify the structure of tasks. Make sure not to overload the users' short- and long-term memories. On average the user is able to remember five things at a time. For example, making a long sequence of menus would be counter-productive for a given task. 
    \item Make sure the task in consistent and provide mental aids for easy retrieval of information from long-term memory. Make sure the user has control over the task. 
    \item Make things visible: bridge the gulfs of Execution and Evaluation. The user should be able to figure out the use of an object by seeing the right buttons or devices for executing an operation. Trying to keep the menus as simple as possible while giving the options necessary, and only those, is a correct implementation of this principle.
\end{enumerate} 


