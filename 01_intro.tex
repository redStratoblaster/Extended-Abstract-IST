\section{Introduction}
\label{sec:intro}

\begin{quotation}
``Man is by nature a social animal; an individual who is unsocial naturally and not accidentally is either beneath our notice or more than human. Society is something that precedes the individual. Anyone who either cannot lead the common life or is so self-sufficient as not to need to, and therefore does not partake of society, is either a beast or a god."
\par - Aristotle.
\end{quotation}

% #############################################################################
\par Nowadays, contact between individuals is increasingly made through digital means: Instant messengers, Voice-over-IP and video calls. Technology allows us to be closer to people, even when they are across the globe. People that do not know each other are brought closer and form social relationships thanks to that same technology. And one particular situation is the approximation of individuals through video games. There are studies that explore the social component of video games and how can players form relationships with each other. For example, studies have shown that cooperative video games can bring families closer \cite{wang_taylor_sun_2018} and improve the social and affective aspects of hospitalized children by having interactions through a video game context with other children in the same condition \cite{gonzalez-gonzalez_toledo-delgado_collazos-ordonez_gonzalez-sanchez_2014}. 
\par Socialization is a key process in the psychological development of a child. As the child gradually becomes a teen, his/her \textbf{social cognition} mastery starts to expand and his/her awareness towards the social environment around him/her increases steadily. If asked how to describe a friend, that same description changes from physical characteristics and tastes (e.g. ``he has brown hair and likes to play football") to more psychological traits, due to him/her starting to perceive the others' actions and behaviours. Also, his/her group of friends shifts from a nebulous semi-structured group of children with whom he/she can have fun and play games to a more heterogeneous and closed group of similar-minded teens that share social activities and opinions \cite{campos_1990}. 
\par As the peer-to-peer relations increase in early pre-adolescence, enthusiast, cooperative and responsive children are usually seen as the more popular in his group of people. However, a child that lacks social interactions, or that feels vulnerable by this psycho-social development and isolates himself, more often will be deem less popular and this can generate anxiety and will diminish the child's self-esteem \cite{tavares_pereira_gomes_monteiro_gomes_2007} \cite{campos_1990} creating a snowball into social alienation. So, a child/pre-adolescent that has isolated himself, either by unconscious self-imposition or due to reasons external to him, has diminished social capabilities and lacks will-power and/or opportunities to engage in social activities with others. 

\par \textbf{How can we help improve pre-adolescents' abilities to establish successful relations with their peers?} To address this issue we explore the use of a Serious Game, but given the complexity of the problem, both in terms of development and evaluation, we will start with a first step in this direction by focusing on how to create such game. \textbf{How can we make such Serious Game fun, enjoyable, and easy to use for children?}


%##############################################################################
\subsection{Approach}

\par We present a proposal and implementation for a serious game, LINA, with a very strong social component that aims to help children improve relations with their peers. In LINA, the players, children from 10 to 12 years old, will try to find out what happened to a missing colleague - Lina - and her story through the discovery of clues and surpassing challenges.
\par There are three concept pillars used in the concept described above: Contact Theory, Augmented Reality and User-Centered Design. Contact Theory tries to end negative conceptions about intergroup peers. We will be applying Contact Theory when trying to get the children to cooperate to achieve a common goal: finding out what happened to Lina. To do so, they will need to complete challenges as a group, by socializing with each other, thereby trying to improve their relationships, and hopefully creating new ones. 
\par Augmented Reality is not a new technology employed in the videogame industry. Recently, it has found enormous popularity with the worldwide-phenomenon that was \textit{Pokémon Go}. Tateno et al. have inclusively written about the hypothesis that the videogame could help children and teens with severe social withdrawal, although studies were not made to support the claim.\cite{tateno_skokauskas_kato_teo_guerrero_2016}
\par As this project will be targeting pre-adolescent children, special care must be considered when designing the concept prototype, as one of our objectives is trying to create a game that children won't find difficult to use. A clean, unambiguous interface with established conventions (like using well-know metaphors for buttons) and restricted freedom makes the users' choice-making clearer and streamlined. However, such concept is bound to iterative revisions to improve the users' interaction.
