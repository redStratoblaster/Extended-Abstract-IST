\section{Conclusions}
\label{sec:concl}
\par An initial prototype for a cooperative Augmented Reality mobile game was presented, with the goal of improving relations between classmates, based on Contact Theory. This game consists in having a sort of interactive theater play, directed by the game, to find what happened to a colleague of theirs through the search of Augmented Reality clues and challenges to be completed through teamwork. Three iterations were made: a paper prototype, a basic technical proof-of-concept involving a server, and a second prototype, more focused on the User Interaction and the challenges proposed to the players.
\par A workshop was conducted to evaluate the Usability and Enjoyment of the prototype through a quantitative questionnaire based on the System Usability Scale, Intrinsic Motivation Inventory, and Game User Experience Satisfaction Scale questionnaires, plus an interview for qualitative feedback. This workshop had some interesting findings and, as it was the first time the game was played by the actual target group, was extremely useful for observing their natural reaction to the whole premise, interface, and challenges proposed. It was successful in two ways: first, it showed that the children are interested and would enjoy playing LINA, and second, it showed that not only is the prototype almost totally usable, but also pointed out what aspects to improve for that end. 
\par With the feedback from the session, future work will consist in improving some aspects of this prototype, while also looking at new challenges and exploring further the technological limits of Augmented Reality in the context of this game.